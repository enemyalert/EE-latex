\documentclass{article}
\usepackage[utf8]{inputenc}
\usepackage{amsmath, amsfonts, amssymb, amsthm}
\usepackage{graphicx, float}
\usepackage{textcomp} %for currency
\graphicspath{{images/}}

% ---- docx customization ----

% for margin
\usepackage[a4paper, top=1.0in, bottom=1.0in, left=1.20in, right=1.0in, heightrounded]{geometry}

% for spacing
\renewcommand{\baselinestretch}{2}

% for line indent
\setlength{\parindent}{0pt}

% for paragraph spacing
\setlength{\parskip}{0.8em}

%text wideness
\setlength{\textwidth}{6in}


% ---- Start ng document :) ----
\begin{document}
\begin{center}

    
\section{Power Systems Review Problems by Powerline}
\subsection{Part 1}
\begin{itemize}

    \item\textbf{Problem 1}\\
    \textbf{ACSR means?}

    Answer: Aluminum conductor steel reinforced
    
    \item\textbf{Problem 2}\\
    \textbf{Which of the following standard transmission system voltages is classified as High Voltage?}

    Answer: 230 kV
    
    \item\textbf{Problem 3}\\
    \textbf{Which of the following voltages is NOT a standard distribution voltage?}

    Answer: 16 kV
    
    \item\textbf{Problem 4}\\
    \textbf{What is the recommended horizontal spacing of three phase conductors for a 34.5 kV transmission line?}

    Answer: 4 ft
    
    \item\textbf{Problem 6}\\
    \textbf{A double circuit 3-phase transmission line has a horizontal spacing of 6.0 ft. and a conductor vertical spacing of 3.0 ft. Calculate the GMD of the parallel lines.}

    Answer: 4.94 ft
    
    \item \textbf{Problem 17}\\
    \textbf{A short 3-phase, 3-wire, transmission line has a receiving end voltage of 4160 volt phase to neutral and serving a balanced 3-phase load of 998400 Volt-Amp at 0.82 p.f lagging. At the sending end, the voltage is 4600 volts phase to neutral and the power factor is 0.77 lagging. What is the series impedance?}

    Answer: 1.635+j6.92 $\Omega$

    \item \textbf{Problem 18}\\
    \textbf{The voltage current and power factor at the input of a single phase transmission line are 2400 volts, 30 amp, and 75 percent respectively. The entire load is connected at the line output or line end, 7 miles from source end. Calculate the voltage at the end in volts. The line resistance is 1.1 $\Omega$ per mile of wire and the reactance is 0.8 $\Omega$ per mile of wire?}

    Answer: 1827

    \item \textbf{Problem 19}\\
    \textbf{A short 230 kV transmission line has an impedance of 5cis(78) $\Omega$. The sending end power is 100 MW at 230 kV and 85\% power factor. What is the voltage at the other end?}

    Answer: 228.2 kV

    \item\textbf{Problem 20}\\
    \textbf{A three phase transmission line has a resistance of 10 $\Omega$ and reactance of 80 $\Omega$ per wire. The load current is 90 amperes and the power factor of the load is 80\% lagging. The sending (generator) end voltage in the line is 44 volts line to line. What is the receiving end voltage?}

    Answer: 34.3 kV

    \item\textbf{Problem 21}\\
    \textbf{A 3-phase 60 Hz transmission line delivers 20 MVA to a load at 66 kV at 80\% p.f lagging . The total series impedance of each line is 15 +j75 $\Omega$. If nominal "pi" circuit is used, what would be the transmission efficiently if the admittance is j6$\times 10^{-4}\ m\Omega$?}

    Answer: 92.6\%

    \item\textbf{Problem 22\\
    A 230 kV transmission line is sending 100 MW power at 230 kV at 90\% power factor. The impedance is 5+j20 $\Omega$ per phase and its capacitive reactance is 2500 $\Omega$. Determine the \% VR of the line.}

    Answer: 3.63\%

    \item\textbf{Problem 23\\
    Calculate the sag of an overhead distribution line having a span of 300 ft between level supports. The conductor is 4/0 copper weighs 2442 $\frac{lbs}{mile}$. The ultimate strength of conductor is 14050 lbs and safety factor is 5.0}

    Answer: 1.85 ft.

    \item\textbf{Problem 24\\
    In a certain circuit analysis, the base chosen are: 34.5 kV and 100 MVA. What is the impedance base?}

    Answer: 11.9 $\Omega$

    \item\textbf{Problem 25\\
    A 69 kV/13.8 kV, 7.5 MVA transformer has 8\% impedance. What is its impedance at base 100 MVA?}    

    Answer: 106.7\%

    \item\textbf{Problem 26, unsure\\
    A 50 MVA, 33 kV/11kV, three phase, wye-delta connected transformer has 3$^{rd}$ impedance. What is the per cent impedance at 100 MVA base and 34.5 kV base?}

    Answer: 5.49\%

    \item\textbf{Problem 27\\
    At a 34.5 kV substation the available fault current is 10 p.u. What is the available fault MVA if the base MVA is 50?}

    Answer: 500

    \item\textbf{Problem 28, unsure\\
    The available fault duty of a certain point of electrical system is 950 MVA at 230 kV, the Thevenin's equivalent impedance is 2.63\%, what is the available fault current}

    Answer: 91 kA

    \item\textbf{Problem 29\\
    In a short circuit study, the positive, the negative, and zero sequence impedance are 0.25 p.u., 0.3 per  and 0.3 p.u. respectively. The base MVA is 100. Determine the fault current for a three phase fault at 115 kV level.}

    Answer: 2 kA

    \item\textbf{Problem 30\\
    At a certain location in an electrical system, the available fault MVA is 500 MVA installed at that location. Determine the short circuit MVA at the secondary side of transformer.}

    Answer: 188 MVA

    \item\textbf{Problem 31\\
    Behind certain point in a system network the equivalent Thevenin's impedance is 0.2 p.u at 100 MVA base. A 115 kKV/34.5 kV, 10 MVA transformer of 5\% impedance is tapped  at this point. If a three phase fault should occur at the secondary of the transformer, what is the maximum fault current?}

    Answer: 2390 amperes

    \item\textbf{Problem 32\\
    A small factory is tapped at 13.8 kV line where the available fault MVA is 150, the line from tapping point to the 3-333 kVA transformer has an impedance of 0.5 $\Omega$. The impedance of each transformer is 4\% and the secondary voltage is 230 volts, what is the approximate three phase fault current at the secondary side of the transformer bank?}

    Answer: 50 kA

    \item\textbf{Problem 33\\
    The transformer to serve a customer is rated 5 MVA, 13.8 kV/480V and its impedance is 5\%. The cable connecting the breaker to the transformer has an impedance of 0.032 $\Omega$ per phase. What is the fault current if a three phase fault occurs at the breaker?}

    Answer: 8000 amperes

    \item\textbf{Problem 34\\
    At a certain point on 34.5 kV network the Thevenin's equivalent sequence is $X_1=j0.15$ per unit at 50 MVA base, $X_0=j0.55$ per unit at 50 MVA base. Find the short circuit current for phase to phase fault at this point.}

    Answer: 4800 ampere

    \item\textbf{Problem 35\\
    At a certain point of the system network the positive, negative, and zero sequence impedances are 0.25, 0.25, 0.3 p.u. respectively. The base MVA is 100. The voltage level at that point is 34.5 kV. Determine the zero sequence current for a line to ground fault.}

    Answer: 2091 A

    \item\textbf{Problem 36\\
    At a certain point of a power system network the positive, negative, and zero sequence impedances are 0.25 p.u., 0.25 p.u., and 0.30 p.u. respectively. The base MVA is 100. The voltage level at that point is 34.5 kV. Determine the short circuit current for a double line to ground fault.}

    Answer: 5906 Amp

    \item\textbf{Problem 37\\
    The indoor 3-phase power center is to be served at 13.8 kV, the power center will include a high side (primary) circuit breaker, a 1500 kVA, 13800/460 volts transformer with 4\% impedance, and the main circuit breaker. If the service point has a short circuit capacity of 900 MVA. What is the momentary and interrupting duty at 3 cycles of the main secondary circuit breaker?}

    A. 56 kA \& 45 kA

    \item\textbf{Problem 38\\
    Which of the following is not one of the classes of surge arresters?}

    Answer: Transmission class

    \item\textbf{Problem 39\\
    What arrester nominal rating shall be used in a 34.5 kV grounded system?}

    Answer: 37 kV

    \item\textbf{Problem 40\\
    What arrester nominal rating shall be used in a 13.8 kV grounded system?}

    Answer: 15 kV  

    \item\textbf{Problem 41\\
    What major integral equipment between AC system and a DC system?}

    Answer: Synchronous converter

    \item\textbf{Problem 42\\
    The cause of nearly all high voltage flashover in transmission lines are due to one of the following. Which one is this?}

    Answer: Dust and dirt

    \item\textbf{Problem 43\\
    Corona occurs when the potential of the conductor in air is raised to such a value that the electric strength of the surrounding air is exceeded. Which of the following statements is NOT correct?}

    Answer: Heavy winds do not have any effect on the critical voltage

    \item\textbf{Problem 44\\
    A power customer draws power at 220 volts from a transformer on a pole. A current transformer with ratio 1200/5 and potential transformer with ratio 1000:1 are used to meter the electric usage. What is the power indicated if the wattmeter reads 800 watts.}

    Answer: 192 MW

    \item\textbf{Problem 45\\
    The CT ratio and PT ratio are 240 and 2000 respectively. The impedance of the transmission line is 10 $\Omega$, an impedance relay is installed to protect the line. What is the impedance of the line as seen by the impedance relay?}

    Answer: 1.2 $\Omega$

    \item\textbf{Problem 46\\
    A device which monitor and operates when certain level has been reached.}

    Answer: Relay

    \item\textbf{Problem 47\\
    It is a protective relay which compares the sum of incoming currents against the sum of the outgoing currents. It operates when there is unbalance.}

    Answer: Differential relay

    \item\textbf{Problem 48\\
    A circuit is disconnected by isolators when}

    Answer: There is no current in time

    \item\textbf{Problem 49\\
    When a lightning wave arrives at an open end transmission line. What happens?}

    Answer: It doubles at that point

    \item\textbf{Problem 50\\
    What is the meaning of SCADA?}

    Answer: Supervisory Control and Data Acquisition

    \item\textbf{Problem 51\\%1
    What is the GMR of seven strands conductor of radius 1 mm of an individual strand?}

    Answer: 2.1 mm

    \item\textbf{Problem 52\\%2 
    A 115 kV line has a horizontal configuration. The distance between adjacent conductor is 9 ft. What is the geometric mean distance of the line?}

    Answer: 11.34 ft.

    \item\textbf{Problem 53\\%3
    A double circuit 3-phase line are arranged at the vertices of a regular hexagon 20 ft on sides. Calculate the equivalent reactance per mile of the parallel line . Conductor has GMR of 0.0403 ft.}

    Answer: 0.368

    \item\textbf{Problem 54\\%4
    A 230 kV, 20 mile transmission line has two bundled conductor per phase, spaced 12 inches apart. The conductor used in the bundle is 336,400 circular mils has a GMR of 0.0244 ft. What is the GMR of the line}

    Answer: 0.156 ft.

    \item\textbf{Problem 55\\%5
    A 115 kV double circuit 3-phase transmission line composed of 336.4 MCM ACSR with GMR of 0.0244 ft. has a horizontal spacing of 18 ft. and a conductor verical spacing of 9.0 ft. Calculate the GMR of the parallel lines.}

    Answer: 0.744 ft. 

    \item\textbf{Problem 56\\%6
    A single phase secondary line has spacing of 12 cm and with a length of 250 meters. The conductor is No. 8 copper with a GMR of 1.27$\times10^{-3}$ m and resistance of 2.36 $\Omega$ per km. Determine the inductive resistance of the line.}

    Answer: 0.17 $\Omega$

    \item\textbf{Problem 57\\%7
    A 5 km long , 3- phase 34.5 kV line has a horizontal configuration of 4 ft. spacing. The conductor is 336.4 MCM ACSR with GMR of 0.0244. What is the inductance of the line?}

    Answer: 5.33 mH

    \item\textbf{Problem 58\\%8
    A transmission line has a triangular configuration of 4ft. spacing. The conductor is 336.4 MCM ACSR. The outside diameter of 336.4 MCM ACSR is 0.721 in. If the length of the line is 30 km, what is the shunt capacitive reactance per phase?}

    Answer: 7782

    \item\textbf{Problem 59\\%9
    The capacitive reactance of a 100 km, 34.5 kV line is 200 k$\Omega$ per km, what is the total capacitance reactance of the line?}

    Answer: 2000 ohms

    \item\textbf{Problem 60\\%10
    The capacitive reactance of a 100 km, 23 kV line is 1200 ohms, what is the capacitance per km?}

    Answer: 2.2$\times10^{-8}$ farad

    \item\textbf{Problem 61\\%11
    The inductive reactance of the line is 0.25cis(35) per km. What is the total reactance of the line at 20 km?}

    Answer: 5cis(35)

    \item\textbf{Problem 62\\%12
    A 5 mile line three phase line has an equilateral spacing of 4 ft. The conductor has GMR of 0.01688 ft. and resistance of 0.303 $\Omega$ per mile. What is the impedance?}

    Answer: 1.515 + j3.32

    \item \textbf{Problem 63\\%13
    A 5 km long, three phase line has a horizontal configuration of 4 ft. spacing. The conductor is 336.4 MCM ACSR with GMR of 0.0244 ft. and  a resistance of 0.306 $\Omega$ per mile. What is the impedance?}

    Answer: 2.22 cis(65)

    \item\textbf{Problem 64\\%14
    A three phase transmission line, 15 km long serves a substation rated 15 MVA at 34.5 kV, 60 Hz. The line impedance is 0.120 + j0.457 $\Omega$ per km. What should be the sending end voltage so that the transformer can be fully loaded at 70\% pf lagging at its rated voltage?}

    Answer: 37200 V

    \item\textbf{Problem 65\\%15
    A short 230 kV transmission line has on impedance of 5cis(78) $\Omega$. The sending end power is 100 MW at 230 kV and 85\% power factor. What is the line losses?}

    Answer: 272 kW

    \item\textbf{Problem 66\\%16
    A 20 miles 3-phase transmission line is to deliver 20,000 kW at 69 kV at 85\% power factor. The line is composed of 300 MCM ACSR conductor, resistance is 0.342 $\Omega$ per mile and GMR of 0.023 ft., and spaced horizontally 8 ft. apart. What is the sending end voltage to neutral?}

    Answer: 42.5 kV

    \item\textbf{Problem 67\\%17
    Each phase of short 3-phase transmission line has an impedance of 15 + j20 $\Omega$. The impressed emf between the line conductors is 13200 volts. The load current connected to this line is balanced takes 1000 kW at lagging power factor. The current per conductor is 70 amperes. What is the receiving end line voltage?}

    Answer: 10.27 kV

    \item\textbf{Problem 68\\%18
    A short 230 kV transmission line has an impedance of 5cis(78) $\Omega$. The sending end power is 100 MW at 230 kV and 85\% power factor. What is the p.f. at the other end?}

    Answer: 85.4\%

    \item\textbf{Problem 69\\%19
    A short 230 kV transmission line has an impedance of 5cis(78) $\Omega$. The sending end power is 100 MW at 230 kV and 85\% power factor. What is the efficiency?}

    Answer: 99.73\%

    \item \textbf{Problem 70\\%20
    A short 230 kV transmission line has an impedance of 5cis(78) $\Omega$. The sending end power is 100 MW at 230 kV and 85\% power factor. What is the percent regulation of the line?}

    Answer: 0.77\%

    \item\textbf{Problem 71\\%21
    A three phase 230 kV transmission line has a series impedance of 3 +j5 $\Omega$ and a shunt reactance of 2500 $\Omega$. It delivers a load of 98750 kW at 222.2 kV with 80\% power factor lagging. Solve for the sending end power.}

    Answer: 99592 kW

    \item\textbf{Problem 72\\%22
    A 230 kV transmission line is sending 100 MW power at 230 kV at 90\% power factor. The impedance is 5 + j20 and its capacitive reactance is 2500 $\Omega$. Determine the receiving end voltage.}

    Answer: 222.83 kV

    \item\textbf{Problem 73\\%23
    A 230 kV transmission line has an impedance of 50cis(78) $\Omega$ and a capacity reactance of 1.2 $\Omega$. It transmit the power of a base load plant. On a certain dry season the sending end power is 100 MW at 235 kV and 95\% power factor continuously for a period of one month. If cost of generation is ₱ 1.30/kWhr. What is the cost of the line losses for one month period?}

    Answer: ₱2.2 million

    \item\textbf{Problem 74\\%24
    The ABCD constants of a 60 Hz. 3 phase long transmission lines are as follows:}    
        \begin{align}
            A &= D = 0.877\angle 1.57^{\circ}\\
            B &= 191.62\angle 79.1^{\circ}\\
            C &= 0.0012\angle 90.4^{\circ}
        \end{align}
    \textbf{This supplies 100 MW load at 230 kV with 90\% power factor. What is the sending voltage?}

     Answer: 269 kV

    \item\textbf{Problem 75\\%25
    In transmission lines the cross arms are made of?}

    Answer: Steel

    \item\textbf{Problem 76\\%26
    Transmission line insulators are made of?}

    Answer: Porcelain

    \item\textbf{Problem 77\\%27
    The use of strain type insulator is made where the conductor are}

    Answer: Any of the above

    \item\textbf{Problem 78\\%28
    A guy wire}

    Answer: Supports the pole

    \item\textbf{Problem 79\\%29
    Calculate the maximum span between level supports for 500 MCM ACSR conductor weighs 4122 lbs/mile if the allowable sag is 2 ft. The ultimate strength of conductor is 24400 lbs and safety factor is 4.0 .}

    Answer: 350 ft.

    \item\textbf{Problem 80\\%30
    A span of 300 m between level supports is expected to have a maximum sag of 12 m when the wind pressure is 12.2 gm/cm$^2$ of projected area. The circular copper conductor has an area of 1.29 cm$^2$ and weighs 1.13 kg/m. If the conductor has a breaking strength of 4.220 kg/cm$^2$. What is the safety factor under these conditions?}

    Answer: 3

    \item\textbf{Problem 81\\%31
    In a certain circuit analysis, the bases chosen are 69 kV and 100 MVA. What is the impedance base?}

    Answer: 47.6 $\Omega$

    \item\textbf{Problem 82\\%32
    The percent impedance of the line is 6\% at 34.5 kV and 100 MVA bases. What is the ohmic impedance?}

    Answer: 0.72

    \item\textbf{Problem 83\\%33
    The impedance of a line is 5\% on 115 kV and 100 MVA bases. What shall be at 120 kV and 10,000 kVA bases?}

    Answer: 0.46\%

    \item\textbf{Problem 84\\%34
    Find the ohmic value of the impedance 3.8\% + j15.2\%. The base values are 100 MVA and 115 kV respectively.}

    Answer: 5 + j20

    \item\textbf{Problem 85\\%35
    The impedance of a transmission line is 30$\Omega$. What is the per unit impedance of 115 kV and 100 MVA bases?}

    Answer: 0.22

    \item\textbf{Problem 86\\%36
    Which of the following is the likely value of the transient reactance of a 100 MW generator?}

    Answer: 10\%

    \item\textbf{Problem 87\\%37
    In a certain line, the positive and zero sequence reactance of 3\% and 15\% respectively. What is the negative sequence reactance?}

    Answer: 3\%

    \item\textbf{Problem 88\\%38
    A 15 MVA, 34.5 kV/6.24kV transformer is connected to infinite bus. The percent impedance of the transformer is 2.5\%. What is the current at 34.5 kV side for a three phase short at the 6.24 kV side?}

    Answer: 10 kiloamperes
    
    \item\textbf{Problem 89\\%39
    At a certain location in an electric system, the available fault MVA is 400 MVA. A 15 MVA, 34.5 kV/6.24 kV, 2.5\% impedance, wye-wye grounded transformer is installed at that location. Determine the short circuit MVA at the secondary.}

    Answer: 240 MVA

    \item\textbf{Problem 90\\%40
    There was a 3-phase fault at a certain point in a 13.8 kV network where the thevenin's equivalent impedance is $\frac{1}{2}\ \Omega$ per phase. What is the magnitude of the fault current?}

    Answer: 15900 A

    \item\textbf{Problem 91\\%41
    There was a phase to phase fault at 13.8 kV system where the thevenin's equivalent impedance is 2.63\%. What is the magnitude of the fault current? Base MVA is 10.}

    Answer: 13800 amperes

    \item\textbf{Problem 92\\%42
    A 5 MVA, 13.8 kV/480 V, 5\% impedance transformer is tapped at 13.8 kV line where the thevenin's equivalent impedance is $\frac{1}{2}\ \Omega$. Determine the fault current at the primary for a three-phase fault at the secondary.}

    Answer: 3.3 kA

    \item\textbf{Problem 93\\%43
    A 10 MVA, 13.8 kV/480 V, 5\% impedance, wye grounded-delta secondary transformer serves an industrial customers. The phase a conductor on a secondary side accidentally touches a grounded point. What is the fault current?}

    Answer: 0

    \item\textbf{Problem 94\\%44
    A utility supplies an industrial plant at 13.2 kV from a 20 MVA transformer whose impedance is 8\%. The utility short circuit capacity at the primary of the transformer is 500 MVA. It is desired to add 3-phase current limiting reactors the secondary of the transformer to limit the initial fault capacity form the utility to 133 MVA. What is the reactance of the reactor required?}

    Answer: 0.265 $\Omega$

    \item\textbf{Problem 95\\%45
    The connected electrical load of an office building is 300 kVA. The main circuit breaker to be installed is 1600 amperes, 2 poles, 250 volts. The Meralco distribution transformer is rated 20,000/230 volts. It is to serve this load with a rated connected load of 500 kVA, single-phase, 60 Hz, 3\% impedance, 230 volts. What interrupting rating should be required for the main circuit breaker?}

    Answer: 75 kA

    \item\textbf{Problem 96\\%46
    Surge arresters are needed in transmission line for the following purposes. Which is the important?}

    Answer: Protect the system from high voltage transient

    \item\textbf{Problem 97\\%47
    What arrester nominal rating shall be used in a 13.8 kV ungrounded system?}

    Answer: 15 kV

    \item\textbf{Problem 98\\%48
    The distribution system is 34.5 kV ungrounded. Which arrester shall be installed to protect a distribution transformer?}

    Answer: 35 kV

    \item\textbf{Problem 99\\%49
    For which of the following equipment current rating is not necessary?}

    Answer: Isolator

    \item\textbf{Problem 100\\%50
    Distance relay is used to?}

    Answer: Measure impedance of a line and operates when the measured impedance goes below a certain point.

    \item\textbf{Problem 101\\%51
    What is the relay that can detect overload in the line?}

    Answer: Overcurrent relay

    \item\textbf{Problem 102\\%52
    In transmission lines, the most effective protection against lightning strikes is one of the following. Which one is this?}

    Answer: Overhead wires

    \item\textbf{Problem 103\\%53
    Which of the following does not belong to the protection of a transmission line?}

    Answer: Reverse power relay

    \item\textbf{Problem 104\\%54
    It is computerized data gathering. monitoring, and switching equipment.}

    Answer: SCADA

    \item\textbf{Problem 105\\%55
    The CT ratio and PT ratio are 240 and 2000 respectively. A reactance relay is installed to protect the line. What is the reactance of the transmission line if the reactance as seen by the reactance relay is 1.41 $\Omega$.}

    Answer: 11.75 $\Omega$

    
\end{itemize}
\newpage
next | page
\end{center}
\end{document}